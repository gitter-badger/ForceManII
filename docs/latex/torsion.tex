\begin{DoxyWarning}{Warning}
Energies computed by Forceman\+II for the torsional and improper torsional energetic components may differ from those of other programs. Read on to see why this is.
\end{DoxyWarning}
\section*{Torsion}

Assume that atom 1 is bonded to atom 2; atom 2 is bonded to atom3, and atom 3 is bonded to atom 4. If we look down the 2-\/3 bond the 1-\/2 and 2-\/3 bonds will make an angle. This angle is the torsion angle. Mathematically this can be viewed as the angle between plane comprised of atoms 1,2, and 3 and the plane comprised of the atoms 2,3,4. Let $\vec{r_{21}}$ be a vector parallel to the 1-\/2 bond pointing from atom 2 to atom 1, $\vec{r_{23}}$ be a vector parallel to the 2-\/3 bond pointing from atom 2 to atom 3, and $\vec{r_{34}}$ be a vector parallel to the 3-\/4 bond pointing from atom 3 to atom 4. $\vec{n_1}=\vec{r_{21}}\times\vec{r_{23}}$ is a vector normal to the plane containing atoms 1, 2, and 3 and $\vec{n_2}=\vec{r_{34}}\times\vec{r_{23}}$ is a vector normal to the plane containing atoms 2, 3, and 4. The dihedral angle, $\phi$ is then given by\+: \[ \cos\theta=\frac{\vec{n_1}\cdot\vec{n_2}}{n_1 n_2}. \] Alternatively\+: \[ \vec{n_1}\times\vec{n_2}=\frac{n_1 n_2\sin\theta \vec{n_3}}{n_3} \] where $\vec{n_3}$ is a unit vector perpindicular to $\vec{n_1}$ and $\vec{n_2}$ and with direction consistent with the right-\/hand rule. Since, both of our cross products involved $\vec{r_{23}}$, we know that $\vec{n_3}=\frac{\vec{r_{23}}}{r_{23}}$. Given that $\vec{n_3}$ is a unit vector we may multiply both sides of the previous equation from the right by it to get\+: \[ \sin\theta=\frac{\left(\vec{n_1}\times\vec{n_2}\right)\cdot\vec{r_{23}}} {n_1 n_2 r_{23}} \] Using the fact that $\tan\theta=\frac{\sin\theta}{\cos\theta}$ we also have \[ \tan\theta=\frac{\left(\vec{n_1}\times\vec{n_2}\right)\cdot\vec{r_{23}}} {r_{23}} \]

As far as symmetries of the torsion angle are concerned, we have one degree of freedom, we can read the sequence forward or backward. Presently, we choose to read the sequence in which ever manner makes atoms 2 and atoms 3 show up in number order (order in input). Parameters for torsions are expected to be given in an analogous manner (atom type 2 should be less then atom type 3).

\subsection*{Improper Torsion}

Related to the torsion angle, the improper torsion angle measures the planarity around an sp $^2$ hybridized atom. Instead of being a non-\/branching path i.\+e. 1 is connected to 2, which is connected to 3, which is connected to 4, the atoms in an improper torsion are all bonded to atom 2. It should be noted that there are other conventions for which atom the central atom is (usually that it is atom 3), but we choose it to be atom 2 as this is naturally where it falls when looping over connectivity.

It is also worth noting a few properties of the improper dihedral angle. Assume for a quadruple of atoms \$ a\+\_\+1,a\+\_\+2,a\+\_\+3,a\+\_\+4\$, we always compute our dihedral angle as the angle between the planes formed from the first three atoms, $a_1, a_2, a_3$, and the last three atoms, $a_2, a_3, a_4$. If we swap the atom originally labeled as 2 and the atom originally labeled as 3, which is the differing conventions for where the central atom goes, the angle changes sign. Swapping atom 1 and atom 4 produces the same sign change. However, swapping atoms 1 and 3 produces a different angle than the original, and swapping 3 and 4 produces yet another unique angle. This is a problem as there are three unique angles that can be generated just by interchanging the order in which the atoms are numbered. After a long search, the best information I can find is on \href{http://chempedia.info/info/165388/}{\tt this website}. Basically they suggest that the order of the orbtial atoms is that of their input and the central atom is given third for A\+M\+B\+ER and first for C\+H\+A\+R\+MM. Again, our code requests that the central atom is given second.

In order to ensure we always compute the same angle, regardless of the order in which the atoms are specified in the input file, we compute the torsions based on the atom type; specifically, the atom that has the lowest atom type is the first atom, atom two is always the central atom, atom 3 is the atom with the next lowest atom type, and atom 4 is the atom with the highest atom type. In the event of a tie, we number the atoms with the same atom types in a clockwise fashion around the central atom. This is most easily done by computing the angle between the vector going from the central atom to the unique atom and the vector going from the central atom to the non-\/unique atom. Lowest angle is closest to the unique angle in a clockwise sense 